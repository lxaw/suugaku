\documentclass{article}
\usepackage[utf8]{inputenc}
\usepackage{CJKutf8}
\usepackage{stix}

\newenvironment{question}
    {\begin{center}
        \begin{tabular}{|p{0.9\textwidth}|}
        \hline\\
    }
    { 
    \\\\\hline
    \end{tabular} 
    \end{center}
    }
%
\newenvironment{proof}
%

    


\title{群論}
\author{Lexington Whalen}
\date{January 2023}

\begin{document}


\begin{CJK}{UTF8}{min}


\maketitle

\section{集合と写像}

\textit{特に斬新なことは書かれていませんでした。集合論をある程度知っている方にとってはこれはおそらく読み飛ばしても大丈夫だと思います。}

\section{群の概念}

\paragraph{2.1群の定義}

\begin{question}
例題1:群Gにおいて、$aa=a$ならば、$a=e$(単位元)である。また、$ab=e$ならば、$b=a^{-1}, a=b^{-1}$である。
\end{question}

\begin{proof}
 証明:$aa=a$ならば、両辺に$a^{-1}$を右からかけると$aa\times a^{-1} = a\times a^{-1}$となり$a=e$となります。また、$ab=e$ならば,両辺に$a^{-1}$を左からかけると$a^{-1}\times ab = a^{-1}\times e$となり$eb=a^{1}$となります。もっと整えると求めている$b=a^{-1}$となります。
\qedsymbol{$\blacksquare$}
\end{proof}

\paragraph{2.4部分群\\\\}


群Gの部分集合Hが次の2つの条件を満たすとき、これをGの部分群という:
$$ i) a, b \in H \Righarrow ab \in H$$
$$ ii) a \in H \Rightarrow a^{-1} \in H $$


\begin{question}
例題1:群Gの部分群であるHがGの部分群であるため必要十分な条件は
$$ iii) a,b \in H \Rightarrow ab^{-1} \in H $$
\end{question}

\begin{proof}

\end{proof}

\end{CJK}

\end{document}

