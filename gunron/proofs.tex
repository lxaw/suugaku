\documentclass{article}
\usepackage[utf8]{inputenc}
\usepackage{CJKutf8}
\usepackage{stix}

\title{群論}
\author{Lexington Whalen}
\date{January 2023}

\begin{document}


\begin{CJK}{UTF8}{min}

\maketitle

\section{巡回群}

\begin{question}
\paragraph{2.4.2}
\textit{
\textbf{巡回群の部分群はまた巡回群になる。}
}
\end{question}
\\\\
\begin{proof}
Gを元gによって生成された巡回群とし、HをGの任意の部分群とする。\\\\
もし$H = \langle e \rangle$であれば、Hは元eによって生成された巡回群となるので、この場合では主張は正しい。\\\\
$H \neq \langle e \rangle$の場合を考えましょう。その時、Hの任意の元hを取り、そのhを$h = g^{k}$と表すことができる。ここでkを$h = g^{k} \in H$となる最小の自然数とする。我々が目指しているのは$\langle g^{k} \rangle = H$であることを証明すること。\\\\

Hは群なので、$\langle a^{k} \rangle \subset H$である。次に、Hの任意の元$b = g^{i}$を取る。割り算の定義により、\\
$$ i = kq + r,  0 \leq r < m, q,r \in \mathbb{Z} $$
となる。これを踏まえて、次のように書くこともできる:\\
$$ a^{i} = a^{kq+r}, a^{r} = a^{i} \times (a^{k})^{-q} $$
ここでkは$a^{k}$となる最小の自然数であったので、$r=0$が言える。

よって、$\langle a^{k} \rangle = H$となる。

$$\mdblksquare$$

\end{proof}

\section{同型}
\begin{question}
\paragraph{2.7.1, pg40}
\textit{
\textbf{群Gの自己同型の全体が群をつくることを証明せよ。}
}
\end{question}
\\\\
\begin{proof}

証明:

背景知識として以下のようなことがわかる:
\\\\
群Gの上の1対1の変換$\sigma$が、その乗法を不変するということは、Gの任意の元a,bに対して$ ab = c$ならば$a^{\sigma}b^{\sigma} = c^{\sigma}$、すなわち、$a^{\sigma}b^{\sigma} = (ab)^{\sigma}$となることで、$\sigma$がGからG自身への同型写像であることにほかならない。このような$\sigma$を群Gの自己同型という。

\end{proof}

\end{CJK}

\end{document}

